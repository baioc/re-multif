
Implementação de componente(s) de circuitos digitais síncronos e/ou assíncronos em substratos biológicos através de regulação genética ou por vias de sinalização bioquímica em organismos simples;
replicando um trabalho relacionado ao tema de forma a atestar seu caráter científico em um artigo próprio.


\subsubsection{Tema}

Implementação de componentes de circuitos digitais por regulação genética ou através de vias de sinalização bioquímica.


\subsubsection{Objetivos}

Encontrar o que já existe na literatura sobre a implementação de componentes de circuitos digitais síncronos e assíncronos em substratos biológicos.
Modelar reações metabólicas, químicas e de expressão gênica para simular um circuito combinacional simples no organismo Mycoplasma genitalium ou E. coli.
Replicar um trabalho científico relacionado ao tema, reproduzir seus experimentos e comparar resultados em um artigo próprio.
Implementar um componente digital útil para aplicação em circuitos assíncronos (C gate, buffer com atraso regulável, componente de memória, etc).


\subsubsection{Contribuições}

Prover uma implementação independente do estudo replicado, atestando publicamente o seu caráter científico.


\subsubsection{Metodologia}

?
