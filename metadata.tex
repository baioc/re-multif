% DO NOT EDIT - automatically generated from metadata.yaml

\def \codeURL{https://github.com/baioc/ReBioComp}
\def \codeDOI{}
\def \dataURL{10/10/2019}
\def \dataDOI{}
\def \editorNAME{}
\def \editorORCID{}
\def \reviewerINAME{}
\def \reviewerIORCID{}
\def \reviewerIINAME{}
\def \reviewerIIORCID{}
\def \dateRECEIVED{01 November 2018}
\def \dateACCEPTED{}
\def \datePUBLISHED{}
\def \articleTITLE{[Re] Synthetic Gene Network Oscillator}
\def \articleTYPE{Replication}
\def \articleDOMAIN{Biological Computing}
\def \articleBIBLIOGRAPHY{bibliography.bib}
\def \articleYEAR{2019}
\def \reviewURL{}
\def \articleABSTRACT{Implementação de componente(s) de circuitos digitais síncronos e/ou assíncronos em substratos biológicos através de regulação genética ou por vias de sinalização bioquímica em organismos simples; replicando um trabalho relacionado ao tema de forma a atestar seu caráter científico em um artigo próprio.}
\def \replicationCITE{Purcell, O.; di Bernardo, M.; Grierson, C. S.; Savery, N. J. A multi-functional synthetic gene network: a frequency multiplier, oscillator and switch. PLoS One. 2011 Feb 17;6(2):e16140. doi: 10.1371/journal.pone.0016140. PMID: 21359152; PMCID: PMC3040778.}
\def \replicationBIB{oscillator}
\def \replicationURL{https://www.ncbi.nlm.nih.gov/pmc/articles/PMC3040778/pdf/pone.0016140.pdf}
\def \replicationDOI{https://doi.org/10.1371/journal.pone.0016140}
\def \contactNAME{Mateus Favarin}
\def \contactEMAIL{}
\def \articleKEYWORDS{rescience c, biological computing, bioinformatics, octave, oscillator, digital circuits}
\def \journalNAME{ReScience C}
\def \journalVOLUME{4}
\def \journalISSUE{1}
\def \articleNUMBER{}
\def \articleDOI{}
\def \authorsFULL{Gabriel Baiocchi de Sant'Anna and Mateus Favarin}
\def \authorsABBRV{G.B.D. Sant'Anna and M. Favarin}
\def \authorsSHORT{Sant'Anna and Favarin}
\title{\articleTITLE}
\date{}
\author[1,\orcid{0000-0001-8364-8969}]{Gabriel Baiocchi de Sant'Anna}
\author[1]{Mateus Favarin}
\affil[1]{Federal University of Santa Catarina (UFSC), Florianópolis, Brazil}
