% DO NOT EDIT - automatically generated from metadata.yaml

\def \codeURL{https://github.com/baioc/re-multif}
\def \codeDOI{10.5281/zenodo.3545451}
\def \dataURL{}
\def \dataDOI{}
\def \editorNAME{}
\def \editorORCID{}
\def \reviewerINAME{}
\def \reviewerIORCID{}
\def \reviewerIINAME{}
\def \reviewerIIORCID{}
\def \dateRECEIVED{}
\def \dateACCEPTED{}
\def \datePUBLISHED{}
\def \articleTITLE{[Re] A Multi-Functional Synthetic Gene Network}
\def \articleTYPE{Replication}
\def \articleDOMAIN{Biological Computing}
\def \articleBIBLIOGRAPHY{bibliography.bib}
\def \articleYEAR{2020}
\def \reviewURL{}
\def \articleABSTRACT{Studies in the field of synthetic biology make constant additions to existing repositories of biological circuits, as well as to the theoretical understanding of their capabilities. Although many synthetic gene networks have been demonstrated able to perform computations using biomolecules, until recently the majority of such models were engineered to implement the functionality of single reusable circuit parts. Purcell et al. have proposed a network capable of multiple functions, allowing for the selection of three different behaviours in a programmable fashion. This work provides an open-source reference implementation with which their in silico experiments were replicated.}
\def \replicationCITE{Purcell O, di Bernardo M, Grierson CS, Savery NJ (2011) A Multi-Functional Synthetic Gene Network: A Frequency Multiplier, Oscillator and Switch. PLOS ONE 6(2): e16140.}
\def \replicationBIB{multif}
\def \replicationURL{https://www.ncbi.nlm.nih.gov/pmc/articles/PMC3040778/pdf/pone.0016140.pdf}
\def \replicationDOI{https://doi.org/10.1371/journal.pone.0016140}
\def \contactNAME{Gabriel Baiocchi de Sant'Anna}
\def \contactEMAIL{baiocchi.gabriel@gmail.com}
\def \articleKEYWORDS{rescience c, octave, biological computing, synthetic biology, programmable genetic circuit, frequency multiplier, oscillator, toggle switch}
\def \journalNAME{ReScience C}
\def \journalVOLUME{4}
\def \journalISSUE{1}
\def \articleNUMBER{}
\def \articleDOI{}
\def \authorsFULL{Gabriel Baiocchi de Sant'Anna and Mateus Favarin Costa}
\def \authorsABBRV{G.B. Sant'Anna and M.F. Costa}
\def \authorsSHORT{Sant'Anna and Costa}
\title{\articleTITLE}
\date{}
\author[1,\orcid{0000-0001-8364-8969}]{Gabriel Baiocchi de Sant'Anna}
\author[1]{Mateus Favarin Costa}
\affil[1]{Universidade Federal de Santa Catarina (UFSC), Florianópolis, Brazil}
