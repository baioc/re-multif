
\subsubsection{Abstract}

Brevemente explicar a área da biologia computacional.\\
Descrever em uma ou duas frases o que foi feito no trabalho replicado: \textbf{circuitos digitais osciladores em substrato biológico}.\\
Destacar que faremos a \textbf{replicação de um trabalho\footnote{Alguma das referências \cite{oscillator} ou \cite{clock}} já existente} seguindo a filosofia do \href{http://rescience.github.io/}{ReScience}.\\
Mencionar se foi possível replicar os experimentos, simulações e resultados ou não.\\


\subsubsection{Introduction}

Alguns conceitos básicos sobre biologia sintética e computação biológica:
apenas aqueles fundamentais para o entendimento deste artigo.\\


\subsubsection{Methods}

Explicar a modelagem que utilizarmos, destacando similaridades e diferenças do trabalho original.\\
Mencionar as ferramentas computacionais utilizadas nos experimentos e simulações, também comparando com o original:
provavelmente utilizaremos \textit{Octave} e/ou \textit{Copasi}.\\
Devemos liberar acesso ao código fonte, parâmetros iniciais utilizados, constantes das reações simuladas, arquivos de configuração, dados intermediários gerados e tudo o que for necessário para tornar nosso trabalho também reprodutível, reusável e replicável.\\


\subsubsection{Results}

Analisar os resultados dos experimentos, comparando-os com os do trabalho de referência através de gráficos e estatísticas referentes às simulações.\\


\subsubsection{Conclusions}

Explicar como foi (ou não) possível replicar o artigo original, destacando quaisquer diferenças nos trabalhos.\\
